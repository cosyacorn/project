\documentclass[pdftex,12pt,a4paper]{article}
\newcommand{\HRule}{\rule{\linewidth}{0.5mm}}
\linespread{1.3}
\usepackage{graphicx}
\usepackage[margin=2.75cm]{geometry}
%\usepackage{tikz}
\usepackage{amsmath}
\usepackage{hyperref}
%\usetikzlibrary{decorations.pathmorphing}
%\tikzset{snake it/.style={decorate, decoration=snake}}


\begin{document}



%%%%%%%%%%%%%%%%%%%%%%
%%%%% TITLE PAGE %%%%%
%%%%%%%%%%%%%%%%%%%%%%

\begin{titlepage}
\begin{center}
\pagenumbering{gobble}

\begin{figure}

\centering
\includegraphics[width=30mm]{Crest.png}

\end{figure}


\textsc{\LARGE Trinity College Dublin}\\
\textsc{\Large School of Mathematics}

\HRule \\[0.4cm]
{\huge \bfseries Spin models on random bipartite graphs \\[0.4cm] }

\HRule \\[1.5cm]

\Large \emph{Author:} \\ Shane Harding \\[0.8cm]

\large \emph{Supervisor:} \\ Prof. Mike Peardon

\end{center}

\end{titlepage}



%%%%%%%%%%%%%%%%%%%%%%%%%%%%%
%%%%% TABLE OF CONTENTS %%%%%
%%%%%%%%%%%%%%%%%%%%%%%%%%%%%

\tableofcontents

\newpage
%\pagenumbering{arabic}
%%%%%%%%%%%%%%%%%%%%
%%%%% ABSTRACT %%%%%
%%%%%%%%%%%%%%%%%%%%

\begin{abstract}

This is my abstract.

\end{abstract}


\newpage
%%%%%%%%%%%%%%%%%%%%%%%%
%%%%% INTRODUCTION %%%%%
%%%%%%%%%%%%%%%%%%%%%%%%

\section{Introduction}

This project is centered around doing Ising Model simulations on random bipartite graphs. As such, it is useful to know more the Ising model and graph theory before we get started.

%%%%% ISING MODEL %%%%%

\subsection{The Ising Model}

\subsubsection{What is the Ising Model?}

\subsubsection{Analytical tools used for studying the Ising Model}

\subsubsection{How it is normally studied in serial and parallel simulations}

%%%%% GRAPH THEORY %%%%%

\subsection{Graph Theory}

Graph theory refers to the mathematical study of \emph{graphs}. A graph is a visual representation of set of objects, known as \emph{vertices}. Some pairs of these obects are then connected by links, known as \emph{edges}. If the edges are said to have orientation (if edge $(a, b) \neq (b, a)$, where $a, b$ are vertices), then we call the graph a directed graph. If the edges don't have orientation (if $(a,b)=(b,a)$) then we call the graph an undirected graph. We will deal only with undirected graphs in this report.

For this project we're not going to consider disconnected graphs. That is, graphs where there are no nodes connecting a vertex, or set of connected vertices, to the rest of the graph. Only connected graphs are considered.

All graphs considered will be random, \emph{bipartite} graphs. A bipartite graph is a graph in which we can divide its vertices into two disjoint sets, $A$ and $B$, such that vertices in $A$ are only connected to vertices in $B$, and vice versa. Disjoint means that the two sets have no element in common.

A random graph is a graph where the edges connect vertices at random; there is no pattern or order to how vertices are connected.

\emph{Trivalent graphs} are another class of the graphs we will be dealing with a lot. For a graph to be trivalent it means that ever vertex has exactly three edges connecting it to three other distinct vertices. 

%%%%% MOTIVATION %%%%%

\subsection{Motivation}

%%%%% MPI %%%%%

\subsubsection{Why this is an interesting MPI problem}

%%%%% PHYSICS %%%%%

\subsubsection{Physical uses of these simulations}


%%%%% AIMS %%%%%
\subsection{Aims}

%%%%% SOFTWARE ARCITEC


\newpage
%%%%%%%%%%%%%%%%%%%
%%%%% RESULTS %%%%%
%%%%%%%%%%%%%%%%%%%

\section{Results}



\newpage
%%%%%%%%%%%%%%%%%%%%%%
%%%%% CONCLUSION %%%%%
%%%%%%%%%%%%%%%%%%%%%%

\section{Conclusion}

hello


\end{document}
